\chapter*{Executive Summary}

This is a \LaTeX template for a WPI project report. Feel free to use and distribute this template as needed. This page will discuss how to use this template and present some of its features. 

This template is organized as follows. The header directory contains all content that appears before the content of the report (i.e., title page, abstract, authorship page, acknowledgements page, executive summary, and the table of contents, figure, and tables. The body directory contains the content of the report (i.e., introduction, background, methodology, results, and conclusion). The footer directory contains the content following the body of the report (i.e., appendix, glossary, references). The Report.tex file links the report together, and the ReportFormat.sty file contains the formatting for the report. 

This paragraph will discuss how to manage references. To add a reference, get its bibtex citation and add it to the file 'footer/References.bib'. To change the style of the references, edit the 'bibliographystyle' command in the file 'footer/References.tex'. To cite a source in line, use the 'cite' command \cite{source1}.

This paragraph will discuss how to manage glossary entires. To add an entry, edit the 'footer/glossary/GlossaryEntries.tex'. To reference an entry use the 'gls command -- \gls{latex} --. To reference a capitalized entry, use the 'Gls' command -- \Gls{latex} --. To reference an acronym use the commands 'acrfull', 'acrlong', or 'acrshort' for \acrfull{wpi}, \acrlong{wpi}, \acrshort{wpi}, respectively.


This paragraph will show you how to include a figures and tables. To reference a figure or table use the 'ref' command -- Figure \ref{fig:wpi1}, Figure \ref{fig:wpi23a}, Figure \ref{fig:wpi23b}, and Table \ref{tab:1}

\begin{figure}
    \centering
    \includegraphics[height=1cm]{header/executivesummary/wpi1.png}
    \caption{The main WPI logo.}
    \label{fig:wpi1}
\end{figure}

\begin{figure*}[t!]
    \centering
    \begin{subfigure}[t]{0.5\textwidth}
        \centering
        \includegraphics[height=1cm]{header/executivesummary/wpi2.png}
        \caption{A second WPI logo.}
        \label{fig:wpi23a}
    \end{subfigure}%
    \begin{subfigure}[t]{0.5\textwidth}
        \centering
        \includegraphics[height=1cm]{header/executivesummary/wpi3.jpg}
        \caption{A third WPI logo.}
        \label{fig:wpi23b}
    \end{subfigure}
    \caption{Two more WPI logos.}
    \label{fig:wpi23}
\end{figure*}

\begin{table}
    \centering
    \begin{tabular}{ |c|c|c| }
        \hline
        cell1 & cell2 & cell3 \\ 
        cell4 & cell5 & cell6 \\ 
        cell7 & cell8 & cell9 \\ 
        \hline
    \end{tabular}
    \caption{An example table}
    \label{tab:1}
\end{table}


