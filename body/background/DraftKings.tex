\section{Fantasy Sports and DraftKings}
%Author: Jackson P
Our sponsoring company, DraftKings, is one of the leaders in online fantasy sports. Fantasy sports are a type of online game where participants draft  rosters of professional sports players to compete against other users on a given platform. Entrants earn points based off of the statistical performance of each professional player they draft in their season games. Examples of some standard statistics tracked by fantasy sports include yards gained, strikeouts thrown and free throws made. These points are then totalled up by an algorithm. The entrant with the highest amount of points earn from their respected drafted players wins the game. Fantasy sports often drive fierce competition to see who can draft the best team. As of 2017, there were nearly 60 million fantasy sports users in the United States alone who spend an annual average of \$556 dollars. As such, the market for daily fantasy sports platforms has grown immensely, with DraftKings taking the lead of this new industry. 

DraftKings is a Boston-based online daily fantasy sports provider. Founded in 2012 by Paul Liberman, Jason Robins and Matt Kalish, DraftKings has quickly grown into a billion dollar company \cite{dk}. Their model differs from traditional fantasy sports in that their contests do not span an entire professional sports season but rather only a few games. DraftKings runs an online platform and a mobile application in which users can discover fantasy sports games (called contests) of many types, ranging in sport, entry fee, maximum number of entrants allowed and numerous other characteristics. On these platforms, users select a contest (fantasy game) and pay a fee to enter up front. Similarly to traditional fantasy sports as described above, they can then select a roster of players from the professional sports games that the contest covers. Then the roster locks and the professional players earn points for the user based on their performance in game. Users whose rosters perform better in the contest are then eligible to receive prize money or free entry to other contests. 


%At any given time, DraftKings is running dozens of contests running over 600,000 in the last 4 years alone.
This project's goal is to create a prediction algorithm to correctly label if a DraftKings contest will reach its maximum number of participants. The majority of the contests the company has run over the last 7 years (approximately $90\%$) fill to the maximum number of entries. When a contest does not fill, DraftKings can lose money, as they guarantee an initial amount in prizes at the start of the contest. It would behoove DraftKings to predict whether a contest will fill to its maximum player amount prior to its close. Thus, if a contest is in danger of not filling, the company can direct advertising money at it to drive up entries. Here, the null hypothesis is that a contest will fill and be a "Success". The alternative hypothesis is then that a contest will not fill; this is what we are trying to detect. Table \ref{tab:confmat} shows a labeled confusion matrix for clarification. In this example, if a contest is predicted to fail ($H_{1}$) and it actually fails ($H_{1}$) then that is a ``True Positive'' the total number of which appears in the bottom right cell. If a contest is predicted to fail ($H_{1}$) and it actually succeeds ($H_{0}$) then that is a ``False Positive'' appearing in the top right cell and so on.

\begin{table}
\centering
\begin{tabular}{| c | c | c |}
\hline
 & \textbf{Predict $H_{0}$} & \textbf{Predict $H_{1}$}  \\ 
 & Predict Contest Fills & Predict Contest Fails \\
\hline
\textbf{Actually $H_{0}$} & True Negative & False Positive  \\ 
Contest Fills & & \\
\hline
\textbf{Actually $H_{1}$} & False Negative & True Positive  \\
Contest Fails & & \\
\hline
\end{tabular}
\caption[Problem-Specific Confusion Matrix]{This table is the confusion matrix of possible prediction alignments in our case of binary classification. For this application, the null hypothesis ($H_{0}$) is that a given contest will fill. The alternative ($H_{1}$) is that the given contest does not fill. Since the goal of this work is to detect failing contests, a failing contest is considered to be a positive and a successful contest is considered to be a negative. The prediction is considered ``True'' if the predicted class is the same as the actual class, and ``False'' otherwise.}
\label{tab:confmat}
\end{table}

For DraftKings' purposes, a false negative is considered approximately ten times worse than a false positive. In other words, predicting that a contest will fill and having it fail to fill is about ten times worse than predicting it will fail to fill and it actually fills. Now that we have identified the important aspects of our classification problem, we can begin to explore the intricacies of the dataset that DraftKings provided our team. 
 