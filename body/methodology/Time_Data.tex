



%randy doesn't like bullet lists
% \begin{itemize}
%   \item Time Series Data Processing
%   \item Exponential Model Fitting
%   \item Final Data Set Setup
%   \item Classification Prediction
%   \item Performance Evaluation
% \end{itemize}

\section{Time Series Data Processing}

The total data cleaning and classification occurred over 5 major steps: Time Series Data Processing, Exponential Model Fitting, Final Dataset Setup, Classification Prediction, Performance Evaluation. In the following chapter, we explore each step, detailing the techniques we utilized to generate our results.

\subsection{Data Chunking}

In its original format, the time series data was organized by month. This meant that a single contest could have its data split into multiple files. Knowing that this organization would involve extra load time for processing contests, we split the time series data into chunks based on each series' id. Each chunk consists of approximately 10,000 contests, and were saved with a naming scheme ``chunkN.csv'' where N is a positive natural number. In total, we ended with 65 chunks of data. Reorganizing the data as such ensures that each chunk contained every data point for each of its' assigned contests, which will save time when loading data. Along with the new chunk formatting, we generated a chunk map: A csv that lists each contests' id and the name of the chunk file that contest is found in. 

\subsection{Data Cleaning}

The original time series data format had columns for ``Minutes Remaining'' (minutes remaining in contest) and ``Entries in Last Minute'' (number of entries received in that minute) as shown in Table 3.1. For our purposes, we preferred to instead have the data in the form of ``Minutes Since Start'' (minutes since contest opened) and ``Summed Entries'' (total number of entries since contest opened). To do this, we performed a cumulative summation per contest in chronological order over the number of entries at each minute. 

\begin{table}
\begin{center}
\begin{tabular}{| c | c | c |}
\hline
 \textbf{Contest ID} & \textbf{Minutes Remaining} & \textbf{Entries in Last Minute} \\ 
 \hline
 10486 & 400 & 2 \\  
 \hline
 10486 & 395 & 1 \\
 \hline
 10486 & 394 & 3 \\
 \hline
 \vdots & \vdots & \vdots \\
 \hline
 10486 & 210 & 5 \\
 \hline
 \vdots & \vdots & \vdots \\
 \hline
 10486 & 0 & 4 \\
 \hline
\end{tabular}
\caption{A representative set of fake time series data as it would have appeared in the originally provided file. Note this reflects only a single contest while the actual files included one full months worth of contests.}
\end{center}
\end{table}

We reformatted the time column, calculating the ``Time Since Start'' value by subtracting the current minutes remaining from the maximum minutes remaining. This inverts the numbering so time starts from 0 and increases until the contest closes. 
Additionally, we added a Boolean column for each point to tell whether that point occurs in the last 240 minutes (4 hours) of the contest or not. A value of 1 means the point occurs before 4 hours remaining (Time Remaining $>$ 240), 0 otherwise. This column will be used later to  separate out which part of the time series occurs before there are 4 hours left in the contest (The time we want to make a prediction at). This separation is intended to simulate the data that would be available when we need to predict a contest's success.
Altogether, the structure for each contest is changed from what's seen in Table 3.1 to something more like Table 3.2.

\begin{table}
\begin{center}
\begin{tabular}{| c | c | c | c |}
\hline
 \textbf{Contest ID} & \textbf{Minutes Since Start} & \textbf{Summed Entries} & \textbf{4 Hours Out} \\ 
 \hline
 10486 & 0 & 2 & 1 \\  
 \hline
 10486 & 5 & 3 & 1 \\
 \hline
 10486 & 6 & 6 & 1 \\
 \hline
 \vdots & \vdots & \vdots & \vdots \\
 \hline
 10486 & 190 & 60 & 0 \\
 \hline
 \vdots & \vdots & \vdots & \vdots \\
 \hline
 10486 & 400 & 310 & 0 \\
 \hline
\end{tabular}
\caption{This is the same representative set of fake data after processing ``Minutes Remaining'' into ``Minutes Since Start'' by subtracting the ``Minutes Remaining'' value from the maximum ``Minutes Remaining'' value. ``Entries in Last Minute'' was also transformed into ``Summed Entries'' by taking a cumulative sum of entry values. A fourth ``4 Hours Out'' boolean column was added with value 1 if ``Minutes Remaining'' was $\geq$ 240 and 0 otherwise.}
\end{center}
\end{table}

We also go through each contest and scale ``Minutes Since Start'' and ``Summed Entries'' to 100 by dividing each point by the max value of its column, then multiplying by 100. When we later fit exponential models to the data, this scaled format will ensure that every model is in the same range, holding the predictive power consistent across contests of varying sizes.  The final scaled values from Table 3.2 can be found in Table 3.3.

\begin{table}
\begin{center}
\begin{tabular}{| c | c | c | c |}
\hline
 \textbf{Contest ID} & \textbf{Minutes Since Start} & \textbf{Summed Entries} & \textbf{4 Hours Out} \\ 
 \hline
 10486 & 0 & 0.6452 & 1 \\  
 \hline
 10486 & 1.25 & 0.9677 & 1 \\
 \hline
 10486 & 1.5 & 1.935 & 1 \\
 \hline
 \vdots & \vdots & \vdots & \vdots \\
 \hline
 10486 & 47.5 & 19.35 & 0 \\
 \hline
 \vdots & \vdots & \vdots & \vdots \\
 \hline
 10486 & 100 & 100 & 0 \\
 \hline
\end{tabular}
\caption{This is the same representative fake data set after scaling both ``Minute Since Start'' and ``Summed Entries'' to 100 in order to standardize the range of values in each column.}
\end{center}
\end{table}

% \subsection{Durations}

% The last pre-processing step for time series data is getting the duration of each contest. In subsequent sections, we explain the methods used to predict the model parameters for a given contest. However, in order to then forecast what the model value will be at the time the contest closes, we need to know the final value of ``Time Since Start''. To do this, we go through all contest ID's and get the max time value which is considered the ``Duration'' of the contest. The duration of the example contest from Tables 3.1 - 3.3 is 400 minutes. A csv with each contest ID and its associated duration was then created.

